\documentclass[a4j,dvipdfmx]{jsarticle}

\usepackage[top=20truemm,bottom=20truemm,left=20truemm,right=20truemm]{geometry}
\usepackage{bm}
\usepackage{enumerate}
\usepackage{amsmath,amssymb}
\usepackage{mathtools}
\usepackage{ascmac}
\usepackage{tcolorbox}
\tcbuselibrary{theorems,breakable}
\usepackage{tikz}
\usetikzlibrary{positioning,intersections,calc,arrows.meta,math}


\newcommand{\answer}
{\noindent
\begin{tikzpicture}[scale=0.2, baseline=2.8pt]
\draw (3.3,1) node{\large\textgt{解 答}};
\draw[thick, rounded corners=3pt] (0,0)--(6.5,0)--(6.5,2.2)--(0,2.2)--cycle;
\end{tikzpicture}\;\\}

\newcommand{\supplement}
{\noindent
\begin{tikzpicture}[scale=0.2, baseline=2.8pt]
\fill[gray] (0,0)--(6.5,0)--(6.5,2.2)--(0,2.2);
\draw (3.3,1) node[white]{\large\textgt{補 足}};
\draw[gray] (0,0)--(6.5,0)--(6.5,2.2)--(0,2.2)--cycle;
\end{tikzpicture}\;\\}

\newcommand{\E}{\mathbb{E}}
\newcommand{\var}{\mathrm{var}}
\newcommand{\cov}{\mathrm{cov}}

\title{PRML 1章 演習問題}
\author{mei}
\date{}

\begin{document}
\maketitle

\begin{itembox}[l]{問題 1.1}
\begin{equation}\label{rss}
    y(x,\bm{w}) = \sum_{j=i}^{M}w_j x_j,
    \qquad
    E(\bm{w}) = \frac{1}{2} \sum_{n=1}^{N} \{y(x_n,\bm{w}) - t_n\}^2
\end{equation}
二乗和誤差(\ref{rss})を最小にする係数$\bm{w}=\{w_i\}$が
次の線形方程式の解として与えられることを示す

\begin{equation}
    \sum_{j=0}^{M} A_{ij} w_j = T_i
\end{equation}
ただし、
\begin{equation*}
    A_{ij} = \sum_{n=1}^{N} (x_n)^{i+j},\qquad T_i = \sum_{n=1}^{N}(x_n)^i t_n
\end{equation*}
\end{itembox}

\answer
\begin{equation*}
\begin{split}
    \frac{\partial E}{\partial w_i} &= \sum_{n=1}^{N} (y_n(x_n,\bm{w}) - t_n)
    \frac{\partial}{\partial w_i} (y_n(x_n,\bm{w}) - t_n) \\
    &= \sum_{n=1}^{N} \left( \sum_{j=0}^{M} (x_n)^j w_j - t_n\right) (x_n)^i \\
    &= 0
\end{split}
\end{equation*}
の時に最小になる \\
よって、
\begin{gather*}
    \sum_{n=1}^{N} \left( \sum_{j=0}^{M} (x_n)^j w_j \right) (x_n)^i  =  \sum_{n=1}^{N} (x_n)^i t_n \\
    \sum_{j=0}^{M} \sum_{n=1}^{N} (x_n)^{i+j} w_j = \sum_{n=1}^{N} (x_n)^i t_n
\end{gather*}
となり、(2)を得る。

\supplement
$E$は$w_i$について下に凸な二次関数だから$\frac{\partial E}{\partial w_i} = 0$となる$w_i$が$E$を最小にする

\newpage
\begin{itembox}[l]{問題 1.2}
式(\ref{rss})に正則化項がついた次の誤算関数を最小にする係数$w_i$が満たす式を求める
\begin{equation}
    {\widetilde{E}(\bm{w})} = \frac{1}{2} \sum_{n=1}^{N} \{y(x_n,\bm{w}) - t_n\}^2 + \frac{\lambda}{2} \lVert \bm{w} \rVert ^ 2
\end{equation}
\end{itembox}

\answer
問題 1.1と同様の計算により、
\begin{equation*}
    \begin{split}
        \frac{\partial E}{\partial w_i} 
        &= \sum_{n=1}^{N} \left( \sum_{j=0}^{M} (x_n)^j w_j - t_n\right) (x_n)^i + \lambda w_i\\
        &= 0
    \end{split}
\end{equation*}
よって、
\begin{equation}
    \sum_{j=0}^{M} A_{ij} w_j = T_i - \lambda w_i
\end{equation}
ただし、
\begin{equation*}
    A_{ij} = \sum_{n=1}^{N} (x_n)^{i+j},\qquad T_i = \sum_{n=1}^{N}(x_n)^i t_n
\end{equation*}
が$w_i$が満たすべき線形方程式系になる


\newpage
\begin{itembox}[l]{問題 1.3}
下の表のような箱がある。箱を$p(r)=0.2, p(b)=0.2, p(g)=0.6$の確率でランダムに選び果物を区別せず等確率で取り出す時、
($\mathrm{i}$)りんごを取り出す確率、($\mathrm{ii}$)選んだ果物がオレンジであった時それがgの箱から取り出されたものである確率、
を求める \\

\centering
\begin{tabular}{c|ccc}
    \hline
    箱 & りんご & オレンジ & ライム \\ \hline
    $r$ & 3 & 4 & 3 \\
    $b$ & 1 & 1 & 0 \\
    $g$ & 3 & 3 & 4 \\ \hline
\end{tabular}
\end{itembox}

\answer
($\mathrm{i}$)
\begin{align*}
    p(F=a) &= \sum_{B} p(F=a|B)p(B) \\
    &= p(F=a|B=r)p(B=r) + p(F=a|B=b)p(B=b) + p(F=a|B=g)p(B=g) \\
    &= \frac{3}{10} \cdot \frac{1}{5} + \frac{1}{2} \cdot \frac{1}{5} + \frac{3}{10} \cdot \frac{3}{5} \\
    &= \frac{17}{50}
\end{align*}

($\mathrm{ii}$)
\begin{align*}
    p(B=g|F=o) &= \frac{p(F=o|B=g)p(B=g)}{p(F=o)} \\
    &= \frac{\frac{3}{10}\cdot\frac{3}{5}}{\frac{4}{10} \cdot \frac{1}{5} 
    + \frac{1}{2} \cdot \frac{1}{5} + \frac{3}{10} \cdot \frac{3}{5}} \\
    &= \frac{1}{2}
\end{align*}

\newpage
\begin{itembox}[l]{問題 1.4}
    hoge
\end{itembox}

\answer

\newpage
\begin{itembox}[l]{問題 1.5}
    hoge
\end{itembox}

\answer


\newpage
\begin{itembox}[l]{問題 1.6}
2つの変数$x,y$が独立ならそれらの共分散が0になることを示す
\end{itembox}

\answer
\begin{equation} \label{cov}
    \cov[x,y] = \E_{x,y}[xy] - \E[x]\E[y]
\end{equation}
一方、
\begin{align*}
    \E_{x,y}[xy] &= \sum_{x}\sum_{y} p(x,y)xy
\end{align*}
であり、$x, y$が独立であることから$p(x,y)=p(x)p(y)$とかけるので、
\begin{align*}
    \E_{x,y}[xy] &= \sum_{x}\sum_{y} p(x,y)xy \\
    &= \sum_{x}\sum_{y} p(x)p(y)xy \\
    &= \sum_{x}p(x)x \sum_{y} p(y)y \\
    &= \E[x] \E[y]
\end{align*}
となる。\\これと($\ref{cov}$)式より、
\begin{equation*}
    \cov[x,y]=0
\end{equation*}
が示される。

\supplement
離散変数で考えたけど連続でも同じ

\end{document}