\documentclass[a4j,dvipdfmx]{jsarticle}

\usepackage[top=20truemm,bottom=20truemm,left=20truemm,right=20truemm]{geometry}
\usepackage{bm}
\usepackage{enumerate}
\usepackage{amsmath,amssymb}
% \usepackage{mathtools}
\usepackage[legacycolonsymbols]{mathtools}
\usepackage{ascmac}
\usepackage{tcolorbox}
\tcbuselibrary{theorems,breakable}
\usepackage{tikz}
\usetikzlibrary{positioning,intersections,calc,arrows.meta,math}


\newcommand{\answer}
{\noindent
\begin{tikzpicture}[scale=0.2, baseline=2.8pt]
\draw (3.3,1) node{\large\textgt{解 答}};
\draw[thick, rounded corners=3pt] (0,0)--(6.5,0)--(6.5,2.2)--(0,2.2)--cycle;
\end{tikzpicture}\;\\}

\newcommand{\supplement}
{\noindent
\begin{tikzpicture}[scale=0.2, baseline=2.8pt]
\fill[gray] (0,0)--(6.5,0)--(6.5,2.2)--(0,2.2);
\draw (3.3,1) node[white]{\large\textgt{補 足}};
\draw[gray] (0,0)--(6.5,0)--(6.5,2.2)--(0,2.2)--cycle;
\end{tikzpicture}\;\\}

\newcommand{\E}{\mathbb{E}}
\newcommand{\var}{\mathrm{var}}
\newcommand{\cov}{\mathrm{cov}}

\title{PRML 1章 演習問題}
\author{mei}
\date{}

\begin{document}
\maketitle

\begin{itembox}[l]{問題 1.1}
\begin{equation}\label{rss}
    y(x,\bm{w}) = \sum_{j=i}^{M}w_j x_j,
    \qquad
    E(\bm{w}) = \frac{1}{2} \sum_{n=1}^{N} \{y(x_n,\bm{w}) - t_n\}^2
\end{equation}
二乗和誤差(\ref{rss})を最小にする係数$\bm{w}=\{w_i\}$が
次の線形方程式の解として与えられることを示す

\begin{equation}
    \sum_{j=0}^{M} A_{ij} w_j = T_i
\end{equation}
ただし、
\begin{equation*}
    A_{ij} = \sum_{n=1}^{N} (x_n)^{i+j},\qquad T_i = \sum_{n=1}^{N}(x_n)^i t_n
\end{equation*}
\end{itembox}

\answer
\begin{equation*}
\begin{split}
    \frac{\partial E}{\partial w_i} &= \sum_{n=1}^{N} (y_n(x_n,\bm{w}) - t_n)
    \frac{\partial}{\partial w_i} (y_n(x_n,\bm{w}) - t_n) \\
    &= \sum_{n=1}^{N} \left( \sum_{j=0}^{M} (x_n)^j w_j - t_n\right) (x_n)^i \\
    &= 0
\end{split}
\end{equation*}
の時に最小になる \\
よって、
\begin{gather*}
    \sum_{n=1}^{N} \left( \sum_{j=0}^{M} (x_n)^j w_j \right) (x_n)^i  =  \sum_{n=1}^{N} (x_n)^i t_n \\
    \sum_{j=0}^{M} \sum_{n=1}^{N} (x_n)^{i+j} w_j = \sum_{n=1}^{N} (x_n)^i t_n
\end{gather*}
となり、(2)を得る。

\supplement
$E$は$w_i$について下に凸な二次関数だから$\frac{\partial E}{\partial w_i} = 0$となる$w_i$が$E$を最小にする

\newpage
\begin{itembox}[l]{問題 1.2}
式(\ref{rss})に正則化項がついた次の誤算関数を最小にする係数$w_i$が満たす式を求める
\begin{equation}
    {\widetilde{E}(\bm{w})} = \frac{1}{2} \sum_{n=1}^{N} \{y(x_n,\bm{w}) - t_n\}^2 + \frac{\lambda}{2} \lVert \bm{w} \rVert ^ 2
\end{equation}
\end{itembox}

\answer
問題 1.1と同様の計算により、
\begin{equation*}
    \begin{split}
        \frac{\partial E}{\partial w_i} 
        &= \sum_{n=1}^{N} \left( \sum_{j=0}^{M} (x_n)^j w_j - t_n\right) (x_n)^i + \lambda w_i\\
        &= 0
    \end{split}
\end{equation*}
よって、
\begin{equation}
    \sum_{j=0}^{M} A_{ij} w_j = T_i - \lambda w_i
\end{equation}
ただし、
\begin{equation*}
    A_{ij} = \sum_{n=1}^{N} (x_n)^{i+j},\qquad T_i = \sum_{n=1}^{N}(x_n)^i t_n
\end{equation*}
が$w_i$が満たすべき線形方程式系になる


\newpage
\begin{itembox}[l]{問題 1.3}
下の表のような箱がある。箱を$p(r)=0.2, p(b)=0.2, p(g)=0.6$の確率でランダムに選び果物を区別せず等確率で取り出す時、
($\mathrm{i}$)りんごを取り出す確率、($\mathrm{ii}$)選んだ果物がオレンジであった時それがgの箱から取り出されたものである確率、
を求める \\

\centering
\begin{tabular}{c|ccc}
    \hline
    箱 & りんご & オレンジ & ライム \\ \hline
    $r$ & 3 & 4 & 3 \\
    $b$ & 1 & 1 & 0 \\
    $g$ & 3 & 3 & 4 \\ \hline
\end{tabular}
\end{itembox}

\answer
($\mathrm{i}$)
\begin{align*}
    p(F=a) &= \sum_{B} p(F=a|B)p(B) \\
    &= p(F=a|B=r)p(B=r) + p(F=a|B=b)p(B=b) + p(F=a|B=g)p(B=g) \\
    &= \frac{3}{10} \cdot \frac{1}{5} + \frac{1}{2} \cdot \frac{1}{5} + \frac{3}{10} \cdot \frac{3}{5} \\
    &= \frac{17}{50}
\end{align*}

($\mathrm{ii}$)
\begin{align*}
    p(B=g|F=o) &= \frac{p(F=o|B=g)p(B=g)}{p(F=o)} \\
    &= \frac{\frac{3}{10}\cdot\frac{3}{5}}{\frac{4}{10} \cdot \frac{1}{5} 
    + \frac{1}{2} \cdot \frac{1}{5} + \frac{3}{10} \cdot \frac{3}{5}} \\
    &= \frac{1}{2}
\end{align*}

\newpage
\begin{itembox}[l]{問題 1.4}
    連続変数$x$上で定義された確立密度$p_x(x)$を考える。$x=g(y)$による変換を施した場合$y$についての密度を最大にする位置
    $\widehat{y}$と$x$についての密度を最大にする位置$\widehat{x}$との間に一般には$\widehat{x}=g(\widehat{y})$の関係が成り立たないことを示す。また、線形変換では
    この式が成り立つことも示す。
\end{itembox}


\answer

\newpage
\begin{itembox}[l]{問題 1.5}
    次の式を示す。
    \begin{equation*}
        \var [f] = \E [{f(x)}^2] - {\E[f(x)]}^2
    \end{equation*}
\end{itembox}

\answer
\begin{align*}
    \var[f] &= \E\left[(f(x) - \E[f(x)])^2\right] \\
    &= \int p(x) (f(x) - \E[f(x)])^2 dx \\
    &= \int p(x)f(x)^2 dx - 2\E[f(x)]\int p(x)f(x)dx + \E[f(x)]^2 \int p(x) dx\\
    &= \E[f(x)^2] - 2\E[f(x)]\E[f(x)] + \E[f(x)]^2 \\
    &= \E[f(x)^2] - \E[f(x)]^2
\end{align*}
より示せた。

\supplement
3行目から4行目では$E[f(x)]$の定義と$p(x)$が規格化されていることを用いた。


\newpage
\begin{itembox}[l]{問題 1.6}
2つの変数$x,y$が独立ならそれらの共分散が0になることを示す
\end{itembox}

\answer
\begin{equation} \label{cov}
    \cov[x,y] = \E_{x,y}[xy] - \E[x]\E[y]
\end{equation}
一方、
\begin{align*}
    \E_{x,y}[xy] &= \sum_{x}\sum_{y} p(x,y)xy
\end{align*}
であり、$x, y$が独立であることから$p(x,y)=p(x)p(y)$とかけるので、
\begin{align*}
    \E_{x,y}[xy] &= \sum_{x}\sum_{y} p(x,y)xy \\
    &= \sum_{x}\sum_{y} p(x)p(y)xy \\
    &= \sum_{x}p(x)x \sum_{y} p(y)y \\
    &= \E[x] \E[y]
\end{align*}
となる。\\これと($\ref{cov}$)式より、
\begin{equation*}
    \cov[x,y]=0
\end{equation*}
が示される。

\newpage
\begin{itembox}[l]{問題 1.7}
次の1変数ガウス分布に関する規格化条件を証明する
\begin{equation*}
    \int_{-\infty}^{\infty} \mathcal{N}(x|\mu, \sigma^2) dx = 1
\end{equation*}
\end{itembox}
\answer
\begin{equation*}
    \int_{-\infty}^{\infty} \mathcal{N}(x|\mu, \sigma^2) dx= I
\end{equation*}
とおき、$I^2$を考える。

\begin{align*}
    I^2 &= \int_{-\infty}^{\infty} \mathcal{N}(x|\mu, \sigma^2) dx \int_{-\infty}^{\infty} \mathcal{N}(y|\mu, \sigma^2) dy\\
    &= \int_{-\infty}^{\infty} \int_{-\infty}^{\infty} \mathcal{N}(x|\mu, \sigma^2) \mathcal{N}(y|\mu, \sigma^2) dxdy\\
    &= \frac{1}{2\pi\sigma^2} \int_{-\infty}^{\infty} \int_{-\infty}^{\infty} 
    \exp \left( -\frac{1}{2\sigma^2}(x-\mu)^2 \right) \exp \left( -\frac{1}{2\sigma^2}(y-\mu)^2 \right) dxdy \\
    &= \frac{1}{2\pi\sigma^2} \int_{-\infty}^{\infty} \int_{-\infty}^{\infty} 
    \exp \left( -\frac{1}{2\sigma^2} \left[ (x-\mu)^2 + (y-\mu)^2 \right] \right) dxdy \\
    &= \frac{1}{2\pi\sigma^2} \int_{-\infty}^{\infty} \int_{-\infty}^{\infty} 
    \exp \left( -\frac{1}{2\sigma^2} (x^2 + y^2) \right) dxdy
\end{align*}
最後の行は改めて$x-\mu\rightarrow x, y-\mu\rightarrow y$とおいた。\\
ここで、極座標変換を施すと、
\begin{align*}
    I^2 &= \frac{1}{2\pi\sigma^2} \int_{-\infty}^{\infty} \int_{-\infty}^{\infty}
    \exp \left( -\frac{1}{2\sigma^2} (x^2 + y^2) \right) dxdy \\
    &= \frac{1}{2\pi\sigma^2} \int_{0}^{\infty} \int_{0}^{2\pi}
    \exp \left( -\frac{1}{2\sigma^2} r^2 \right) r drd\theta \\
    &= \frac{1}{\sigma^2} \int_{0}^{\infty} \exp \left( -\frac{1}{2\sigma^2} r^2 \right) r dr \\
    &= - \int_{0}^{\infty} \exp \left( -\frac{1}{2\sigma^2} r^2 \right) \left( -\frac{1}{2\sigma^2} r^2 \right)' dr \\
    &= - \left[ \exp \left( -\frac{1}{2\sigma^2} r^2 \right) \right]_0^{\infty} \\
    &= 1
\end{align*}
よって、$I \ge 0$より$I=1$が示された

\newpage
\begin{itembox}[l]{問題 1.8}
    1変数ガウス分布の平均値が$\mu$で与えられることを示し、次に規格化条件
    \begin{equation*}
        \int_{-\infty}^{\infty} \mathcal{N}(x|\mu, \sigma^2) dx= 1
    \end{equation*}
    の両辺を$\sigma^2$に関して微分することで、ガウス分布の2次のモーメントを求める。
    最後に分散を求める。
\end{itembox}
まず、$\E[x]$を求める。
\begin{align*}
    \E[x] &= \int_{-\infty}^{\infty} \mathcal{N}(x|\mu, \sigma^2) x dx \\
    &= \frac{1}{(2\pi \sigma^2)^{1/2}} \int_{-\infty}^{\infty}\exp \left( -\frac{1}{2\sigma^2}(x-\mu)^2 \right) x dx
\end{align*}
ここで、$x-\mu \rightarrow x$と置き換えると、
\begin{align*}
    \E[x] &= \frac{1}{(2\pi \sigma^2)^{1/2}} \int_{-\infty}^{\infty}\exp 
    \left( -\frac{1}{2\sigma^2}x^2 \right) (x+\mu) dx \\
    &= \frac{1}{(2\pi \sigma^2)^{1/2}} \int_{-\infty}^{\infty}\exp 
    \left( -\frac{1}{2\sigma^2}x^2 \right) x dx
    + \mu \frac{1}{(2\pi \sigma^2)^{1/2}} \int_{-\infty}^{\infty}\exp 
    \left( -\frac{1}{2\sigma^2}x^2 \right) dx \\
    &= C \int_{-\infty}^{\infty}\exp 
    \left( -\frac{1}{2\sigma^2}x^2 \right) \left( -\frac{1}{2\sigma^2}x^2 \right)' dx
    + \mu \int_{-\infty}^{\infty} \mathcal{N}(x|\mu, \sigma^2) dx \qquad (C:\mathrm{Const.})\\
    &= 0 + \mu \cdot 1 \\
    &= \mu
\end{align*}
$\E[x^2]$については、
\begin{align*}
    \int_{-\infty}^{\infty} \mathcal{N}(x|\mu, \sigma^2) dx
    &= \frac{1}{(2\pi \sigma^2)^{1/2}} \int_{-\infty}^{\infty}\exp \left( -\frac{1}{2\sigma^2}(x-\mu)^2 \right) dx \\
    &= 1
\end{align*}
の両辺を$\sigma^2$について微分すると、

\begin{align*}
    -\frac{1}{2}\frac{1}{(2\pi)^{1/2}} \frac{1}{\sigma^3} 
    \int_{-\infty}^{\infty}\exp \left( -\frac{1}{2\sigma^2}(x-\mu)^2 \right) dx
    + \frac{1}{(2\pi \sigma^2)^{1/2}} \int_{-\infty}^{\infty}\exp \left( -\frac{1}{2\sigma^2}(x-\mu)^2 \right)
    \left(\frac{1}{2\sigma^4}(x-\mu)^2 \right) dx \\
    = -\frac{1}{2\sigma^2}\frac{1}{(2\pi \sigma^2)^{1/2}}
    \int_{-\infty}^{\infty}\exp \left( -\frac{1}{2\sigma^2}(x-\mu)^2 \right) dx
    + \frac{1}{2\sigma^4} \frac{1}{(2\pi \sigma^2)^{1/2}} \int_{-\infty}^{\infty}\exp \left( -\frac{1}{2\sigma^2}(x-\mu)^2 \right)
    (x-\mu)^2 dx \\
    = -\frac{1}{2\sigma^2} + \frac{1}{2\sigma^4}(\E[x^2] - 2\mu\E[x] + \mu^2) \\
    = -\frac{1}{2\sigma^2} + \frac{1}{2\sigma^4}(\E[x^2] - \mu^2) \\
    = 0
\end{align*}
よって、
\begin{equation*}
    \E[x^2] = \mu^2 + \sigma^2
\end{equation*}
を得る。
以上より、分散は
\begin{align*}
    \var[x] &= \E[x^2] - \E[x]^2 \\
    &= \mu^2
\end{align*}
と求められる。

\newpage
\begin{itembox}[l]{問題 1.9}
    1変数ガウス分布(\ref{normal})のモードおよび多変量ガウス分布(\ref{multi_normal})のモードが$\mu$や$\bm{\mu}$で与えられることを示す。
    \begin{equation}\label{normal}
        \mathcal{N}(x|\mu, \sigma^2) = 
        \frac{1}{(2\pi \sigma^2)^{1/2}}\exp \left( -\frac{1}{2\sigma^2}(x-\mu)^2 \right)
    \end{equation}
    \begin{equation}\label{multi_normal}
        \mathcal{N}(\bm{x}|\bm{\mu}, \bm{\Sigma}) = \frac{1}{(2\pi)^{D/2}} \frac{1}{|\bm{\Sigma}|^{1/2}}
        \exp \left( -\frac{1}{2} (\bm{x} - \bm{\mu}) ^ {\mathrm{T}}
        \bm{\Sigma}^{-1} (\bm{x} - \bm{\mu}) \right)
    \end{equation}
\end{itembox}
\answer
まず1変数について考える。上に凸なので微分が0になる点で確率分布が最大値を取る
\begin{align*}
    \frac{d}{dx} \mathcal{N}(x|\mu, \sigma^2) &= 
    \frac{d}{dx} \left[ \frac{1}{(2\pi \sigma^2)^{1/2}} \exp \left( -\frac{1}{2\sigma^2}(x-\mu)^2 \right) \right] \\
    &= \frac{1}{(2\pi \sigma^2)^{1/2}} \frac{-1}{\sigma^2} (x-\mu) \exp \left( -\frac{1}{2\sigma^2}(x-\mu)^2 \right) \\
    &= 0
\end{align*}
よって$x=\mu$で$\mathcal{N}$が最大になる、つまりモードが$\mu$で与えられる。\\
次に多変量の場合を考える。指数部分の二次形式を考えると、
\begin{align*}
    -\frac{1}{2} (\bm{x} - \bm{\mu}) ^ {\mathrm{T}}
        \bm{\Sigma}^{-1} (\bm{x} - \bm{\mu}) &= 
        - \frac{1}{2} \sum_{j=1}^{D} \sum_{i=1}^{D} (x_i - \mu_i)^2\cov[x_i, x_j] 
\end{align*}
となるため、勾配が0になるとき
\begin{align*}
    \frac{\partial}{\partial x_k} \mathcal{N}(\bm{x}|\bm{\mu}, \bm{\Sigma}) &= 
    \frac{-1}{2} \sum_{i=1}^{D} 2(x_k-\mu_k)\cov[x_i,x_k] \\
    &=-(x_k-\mu_k) \sum_{i=1}^{D} \cov[x_i, x_k]
\end{align*}
よって、$x_k=\mu_k$の時$\mathcal{N}$が最大になる、つまりモードが$\bm{\mu}$で与えられる

\newpage
\begin{itembox}[l]{問題 1.10}
    2変数が独立の時、それらの和の平均と分散が
    \begin{gather*}
        \E[x+z] = \E[x] + \E[z] \\
        \var[x+z] = \var[x] + \var[z]
    \end{gather*}
    を満たすことを示す
\end{itembox}
まず期待値から
\begin{align*}
    \E[x+z] &= \iint p(x,z) (x+z) dx dz \\
    &= \iint p(x,z)x dx dz + \iint p(x,z)z dx dz \\
    &= \int x \left( \int p(x,z) dz \right) dx + \int z \left( \int p(x,z) dx \right) dz \\
    &= \int p_z(x) dx + \int p_x(z) dz \\
    &= \E[x] + \E[z]
\end{align*}
独立を仮定しなくても周辺分布を考えればよいので線形性は常に成り立つ \\
次に分散の関係式を示す
\begin{align*}
    \var[x+z] &= \E[(x+z)^2] - \E[x+z]^2 \\
    &= \E[x^2] + 2\E[xz] + \E[z^2] - (\E[x] + \E[z])^2 \\
    &= \E[x^2] - \E[x]^2 + \E[z^2] - \E[z]^2 + 2(\E[xz] - \E[x]\E[z]) \\
    &= \var[x] + \var[z] + 2(\E[xz] - \E[x]\E[z]) \\
\end{align*}
2行目で線形性を用いた\\
そして、
\begin{align*}
    \E[xz] &= \iint p(x,z) xz dx dz \\
    &= \iint p(x)p(z) xz dx dy \\
    &= \int xp(x) dx \int zp(z) dz \\
    &= \E[x]\E[z]
\end{align*}
3行目で独立であることから$p(x,y)=p(x)p(y)$を用いた\\
よって、
\begin{equation*}
    \var[x+z] = \var[x] + \var[z]
\end{equation*}
が示せた。
\end{document}