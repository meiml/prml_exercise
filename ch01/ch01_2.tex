\documentclass[a4j,dvipdfmx]{jsarticle}

\usepackage[top=20truemm,bottom=20truemm,left=20truemm,right=20truemm]{geometry}
\usepackage{bm}
\usepackage{enumerate}
\usepackage{amsmath,amssymb}
% \usepackage{mathtools}
\usepackage[legacycolonsymbols]{mathtools}
\usepackage{ascmac}
\usepackage{tcolorbox}
\tcbuselibrary{theorems,breakable}
\usepackage{tikz}
\usetikzlibrary{positioning,intersections,calc,arrows.meta,math}


\newcommand{\answer}
{\noindent
\begin{tikzpicture}[scale=0.2, baseline=2.8pt]
\draw (3.3,1) node{\large\textgt{解 答}};
\draw[thick, rounded corners=3pt] (0,0)--(6.5,0)--(6.5,2.2)--(0,2.2)--cycle;
\end{tikzpicture}\;\\}

\newcommand{\supplement}
{\noindent
\begin{tikzpicture}[scale=0.2, baseline=2.8pt]
\fill[gray] (0,0)--(6.5,0)--(6.5,2.2)--(0,2.2);
\draw (3.3,1) node[white]{\large\textgt{補 足}};
\draw[gray] (0,0)--(6.5,0)--(6.5,2.2)--(0,2.2)--cycle;
\end{tikzpicture}\;\\}

\newcommand{\E}{\mathbb{E}}
\newcommand{\var}{\mathrm{var}}
\newcommand{\cov}{\mathrm{cov}}

\title{PRML 演習問題 1.11~1.20}
\author{mei}
\date{}

\begin{document}
\maketitle

\begin{itembox}[l]{問題 1.11}
    1次元正規分布の$\mu$と$\sigma^2$の最尤解を求める
\end{itembox}
\answer
正規分布から独立に生成された
スカラー変数のデータセット$\mathbf{x}=(x_1,x_2,\cdots,x_N)^{\mathrm{T}}$があったとする。\\
この時対数尤度は
\begin{equation*}
    \ln p(\mathbf{x}|\mu,\sigma^2) = -\frac{1}{2\sigma^2}
    \sum_{n=1}^{N}(x_n-\mu)^2 - \frac{N}{2}\ln \sigma^2
    - \frac{N}{2} \ln(2\pi)
\end{equation*}
\begin{align*}
    \frac{\partial}{\partial \mu} \ln p(\mathbf{x}|\mu,\sigma^2) &=
    -\frac{1}{\sigma^2} \sum_{n=1}^{N}(x_n-\mu) \\
    &= 0
\end{align*}
よって、
\begin{gather*}
    \sum_{n=1}^{N} (x_n-\mu) = 0 \\
    \mu = \frac{1}{N}\sum_{n=1}^{N} x_n
\end{gather*}
となり、$\mu$の最尤解は標本平均と一致する\\
一方、分散については
\begin{align*}
    \frac{\partial}{\partial \sigma^2} \ln p(\mathbf{x}|\mu,\sigma^2) &=
    \frac{1}{2\sigma^4}
    \sum_{n=1}^{N}(x_n-\mu)^2 - \frac{N}{2} \frac{1}{\sigma^2} \\
    &= 0
\end{align*}
よって、
\begin{gather*}
    \sigma^2 = \frac{1}{N} \sum_{n=1}^{N} (x_n - \mu)^2
\end{gather*}
$\mu$の最尤解を$\mu_\mathrm{ML}$と書くと
\begin{equation*}
    \sigma^2 = \frac{1}{N} \sum_{n=1}^{N} (x_n - \mu_{\mathrm{ML}})^2
\end{equation*}
となり、$\mu_{\mathrm{ML}}$は標本平均に一致したから、
$\sigma^2$は標本平均に対する標本分散になる

\newpage
\begin{itembox}[l]{問題 1.12}
    \begin{equation*}
        \E[x_n, x_m] = \mu^2 + I_{nm}\sigma^2 
    \end{equation*}
    を示す。ただし$x_n, x_m$は平均$\mu$、分散$\sigma^2$の正規分布から生成されたデータ点であり、$I_{nm}=\delta_{nm}$である
\end{itembox}
\answer
$(\mathrm{i}) \quad n=m$のとき \\ \\
左辺は二次のモーメントであり、
\begin{equation*}
    \E[x_n^2]=\mu^2+\sigma^2
\end{equation*}
となるので成立 \\ \\
$(\mathrm{ii}) \quad n=m$のとき


\end{document}